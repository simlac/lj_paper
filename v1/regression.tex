In this section we present some regression analysis results for studying the dependence of a community's growth rate, $r_g(\Cc)$, on its percentage of attractive members. As mentioned before, our primary  goal is to determine  factors that affect the growth of a community in the future.  As we will show in this section through our regression analysis, there is a strong  positive correlation between the two.

Remember that the time span between the snapshots is shown in Table \ref{tb:data}. For each snapshot $t\in\{1,\ldots,6\}$ and for each community $\Cc$, we compute the following parameters: the fraction of attractive members of community $\Cc$ at snapshot $t$, i.e., $p_a(t,\Cc)$,  and its growth rate $r_g(t,\Cc)$, In addition to these time variant parameters, for each community we compute its average clustering coefficient $c_i$ \cite{clustering}, average degree $d_i$, and the percentage of \emph{social} people $p_s(\Cc)$ defined as
\begin{align}
p_s(\Cc) = \frac{\sum\limits_{v\in\Cc}s_v}{|\Cc|}.
\end{align}

As the first example, let us consider the following hypothesis,
\begin{align}
r_g(3,\Cc) = \alpha_{1,3}  p_a(1,\Cc) + n_{\Cc},\label{eq:reg1}
\end{align}
which predicts a linear dependence of $r_g(3,\Cc)$ on $p_a(1,\Cc)$.  In \eqref{eq:reg1} , $n_{\Cc}$ represents an independent Gaussian noise distributed as  $n_{\Cc} \sim N(0,\sigma^2)$. The analysis of variance (anova) analysis table generated by R is shown in Table \ref{tb:anova_g3_vs_a1}. The fitted coefficient is $\hat{\alpha}_{1,3}=0.2903$, and the last column of the table shows how significant this coefficient is. In order to give a picture of this linear correlation, Fig.4 shows  growth rates $r_g(3,\Cc)$  versus percentages of attractive members $p_a(1,\Cc)$ in a log scale for the communities which have a positive growth rate.
\begin{figure}
\begin{center}
\includegraphics[width=9cm]{../figures/gr3_vs_att1.pdf}
\caption{$r_g(3,\Cc)$ versus $p_a(1,\Cc)$ for communities with positive growth rate in a log basis}
\end{center}
\label{fig:g5_vs_a1}
\end{figure}

\begin{table}
\caption{Anova table: \texttt{lm($r_g(3,\Cc) \sim p_a(1,\Cc)$)}, fitted coefficient = $0.2903$}
\begin{center}
\begin{tabular}{c|c|c|c|c|c}
    &  Df  & Sum Sq & Mean Sq & F value &   Pr($>$F) \\ \hline
att1     &{\small  1 }& {\small 0.050669} & {\small 0.050669}  & {\small 286.16} &  $<$ {\small 2.2e-16 ***}\\ \hline
Res. & {\small 511} & {\small 0.090479} & {\small 0.000177}  & &
\end{tabular}
\end{center}
\label{tb:anova_g3_vs_a1}
\end{table}





\begin{table}
\caption{Anova table: \texttt{lm($r_g(5,\Cc) \sim p_a(2,\Cc)$)}, fitted coefficient = $0.3130$}
\begin{center}
\begin{tabular}{c|c|c|c|c|c}
         &  Df  & Sum Sq & Mean Sq & F value &   Pr($>$F)    \\ \hline
att2 &    1 & 0.03347 & 0.03347 & 19.058 & 1.574e-05 ***\\ \hline
Res. & 451 & 0.79199 & 0.00176 & \\
\end{tabular}
\end{center}
\label{tb:anova_g5_vs_a2}
\end{table}






Tables \ref{tb:anova_g5_vs_a2} to \ref{tb:anova_g6_vs_a5} show anova tables derived by performing similar analyses for some other snapshots.   In each case the growth rate in one snapshot is regressed against the percentage of attractive members in one of its previous snapshots. More specifically Table \ref{tb:anova_g5_vs_a2} contains anova analysis for testing $r_g(5,\Cc) = \alpha_{2,5} p_a(2,\Cc) + n_{\Cc}$, Table \ref{tb:anova_g5_vs_a3} reflects test results for $r_g(5,\Cc) = \alpha_{3,5} p_a(3,\Cc) + n_{\Cc}$, and Table \ref{tb:anova_g6_vs_a5} shows anova analysis for testing $r_g(6,\Cc) = \alpha_{5,6} p_a(5,\Cc) + n_{\Cc}$. The corresponding fitted coefficients are $\hat{\alpha}_{2,5}=0.3130$, $\hat{\alpha}_{3,5}=0.2624$, and $\hat{\alpha}_{5,6}=0.1651$. It can be observed that the fitted coefficients are in the range of $[1.5,3.5]$, and are  consistent. Moreover, the anova tables in all four cases state the significance of the fitted coefficients. 

One other parameter that can be considered as a potential candidate for predicting the growth of a community $\Cc$ is the clustering coefficient of its  induced graph $\mathcal{G}_{\Cc}$. To check this we test the following hypothesis: $r_g(3,t) = \beta_3  c_{\Cc} + n_{\Cc}$, where again $n_{\Cc}$ represents independent additive white Gaussian noise. The fitted coefficient and the anova analysis results are shown in Table \ref{tb:anova_g3_vs_cc}.   The fitted coefficient is $\hat{\beta}= 0.0130 $, but, according to the table, does not seem to of significance.


One other parameter of potential relevance is the average degree among the members of the community. Remember that we defined a person's sociability in terms of its degree. Therefore, the average degree of a community reflect on average how social its members are. To perform a test to check this dependence, we try $r_g(3,\Cc) = \gamma_3  \bar{d_{\Cc}} + n_i$ where $\bar{d}_{\Cc}$ denotes the average degree in community $\Cc$. The anova analysis is shown in Table \ref{tb:g3_vs_d}. The fitted coefficient is $\hat{\beta}=0.00086$ which again does not have any significance.


Finally as a sanity check we consider predicting the growth rate of a community based on its growth rate in one of the previous snapshots.  As an example of such analysis, Table \ref{tb:anova_g6_vs_g1} shows the anova table generated by R for checking $r_g(3,\Cc) = \eta_{1,3}  r_g(1,\Cc) + n_{\Cc}$ where $n_i\sim N(0,\sigma^2)$. Here the fitted coefficient turns out to be $\hat{\eta}_{1,3}=-0.1408$ which is negative. This shows the difference between the growth rate and percentage of attractive members. While the growth rate at some snapshot fails to predict the growth rate at future snapshots, the percentage of attractive members seems to be a reasonable factor to consider. To demonstrate pictorially how $r_g(1,\Cc)$ fails to predict $r_g(3,\Cc),$ Fig.~\ref{fig:g3_vs_g1} shows $r_g(3,\Cc)$ versus $r_g(1,\Cc)$ for 1111 communities that have $|r_g(1,\Cc)|<0.1$ and $|r_g(3,\Cc)|<0.1$.
\begin{figure}
\begin{center}
\includegraphics[width=90mm]{../figures/gr3_vs_gr1.pdf}\caption{$r_g(3,\Cc)$ versus $r_g(1,\Cc)$}\label{fig:g3_vs_g1}
\end{center}
\end{figure}
 It can be observed that the points are almost uniformly distributed around zero.




\begin{table}
\caption{Anova table: \texttt{lm($r_g(5,\Cc) \sim p_a(3,\Cc)$)}, fitted coefficient = $0.2624$}
\begin{center}
\begin{tabular}{c|c|c|c|c|c}
         &   {\small Df}  &  {\small Sum Sq} &  {\small Mean Sq }&  {\small F value} &   {\small  Pr($>$F)}    \\ \hline
 {\small att3}  &     {\small 1} &  {\small 0.02995} &  {\small 0.02995} & {\small  14.524 }& {\small  0.0001585 ***}\\ \hline
 {\small Res.} &  {\small 431} &  {\small 0.88880 }& {\small  0.00206} &            
\end{tabular}
\end{center}
\label{tb:anova_g5_vs_a3}
\end{table}%


\begin{table}
\caption{Anova table: \texttt{lm($r_g(6,\Cc) \sim p_a(5,\Cc)$)}, fitted coefficient = $0.1651$}
\begin{center}
\begin{tabular}{c|c|c|c|c|c}
         &   {\small Df } &  {\small Sum Sq} & {\small  Mean Sq} &  {\small F value} &  {\small   Pr($>$F)   } \\ \hline
 {\small att5} &         {\small 1}  & {\small  0.13525 }&  {\small 0.13525 }& {\small   96.18 }&  $<$  {\small 2.2e-16 ***}\\ \hline
 {\small Res.} & {\small 1267 }& {\small 1.78164 }&  {\small 0.00141}     &
\end{tabular}
\end{center}
\label{tb:anova_g6_vs_a5}
\end{table}%


\begin{table}
\caption{Anova table: \texttt{lm($r_g(3,\Cc)\sim c_{\Cc}$)}, fitted coefficient = $0.0130$}
\begin{center}
\begin{tabular}{c|c|c|c|c|c}
         &   {\small Df}  &  {\small Sum Sq} &  {\small Mean Sq} &  {\small F value} &   {\small  Pr($>$F)}    \\ \hline
 {\small cc    } &       {\small 1 }&  {\small 0.001064 }&  {\small 0.001064 } & {\small  4.5052  }& {\small  0.03401 *}\\ \hline
 {\small Res.} & {\small  1129} &  {\small 0.266616 }&  {\small 0.000236   }              &
\end{tabular}
\end{center}
\label{tb:anova_g3_vs_cc}
\end{table}%


\begin{table}
\caption{Anova table: \texttt{lm($r_g(3,\Cc)\sim d_{\Cc}$)}, fitted coefficient = $0.00086$}
\begin{center}
\begin{tabular}{c|c|c|c|c|c}
         &   {\small Df } &  {\small Sum Sq} &  {\small Mean Sq} & {\small  F value} &   {\small  Pr($>$F)  }  \\ \hline
 {\small d     }   &   {\small  1} &  {\small 0.000110 }&  {\small 0.000110 }&  {\small 1.0383}& {\small  0.3085 }\\ \hline
 {\small Res. }&  {\small 992 }& {\small  0.105439} & {\small  0.000106  }         &
\end{tabular}
\end{center}
\label{tb:g3_vs_d}
\end{table}%



\begin{table}
\caption{Anova table: \texttt{lm($r_g(3,\Cc) \sim r_g(1,\Cc)$)}, fitted coefficient = $-0.14085$}
\begin{center}
\begin{tabular}{c|c|c|c|c|c}
         & {\small   Df } &  {\small Sum Sq} & {\small  Mean Sq }&  {\small F value} &    {\small Pr($>$F)  }  \\ \hline
 {\small gr1   }&       {\small 1} &  {\small 0.00576 }&  {\small 0.00576}  &  {\small 10.802 }& {\small  0.001045 **}\\ \hline
 {\small Res. }&  {\small 1118} & {\small  0.59572 }& {\small  0.00053       }  &  \\
\end{tabular}
\end{center}
\label{tb:anova_g6_vs_g1}
\end{table}%






