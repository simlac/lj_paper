We have studied six snapshots of about $2000$ communities from the online social network LiveJournal. Our main goal was to find the structural features of communities that can help us with the prediction of growth. We have found that the percentage of \emph{attractive} members of a community in a short time interval, has a strong correlation by its growth rate in the future. We also tried to measure the correlation between the growth and some structural properties of the communities such as  clustering coefficient and average degree. None of them showed evidence of strong correlation.

When we focused on  attractiveness, we found that people are attractive for a short period, and it is not something intrinsic to the community. Also among the newly joined members, the percentage of attractive members is much higher. On the other hand, people are likely to be attractive in different communities that they belong to at the same time.  All these gives us the sense that attractiveness is strongly related to being active.

\subsection{Future Work}

We believe that the percentage of \emph{attractive} members of a community can be a good way to
predict  community growth. The next step is to understand this notion of being \emph{attractive} more 
thoroughly. Does our hypothesis of its being strongly related to activeness of a member is true? 
Can we relate it to some inherent structure of a community? 

Another future work is to come up with a model that predicts the growth of a community using the attractiveness measure. In order to examine the consistency of our results, applying similar analysis to user-defined communities in another social network  is also crucial.