Many studies on online social networks and biological
networks have focused on properties of network evolution such as shrinking 
diameters and edge creation. They mainly study the change in the overall
network over time. We briefly outline some of the work in this area.
Kumar et al.  \cite{kumar_tomkins} studied the evolution of structure within two large network data sets; Flickr and Yahoo! 360. Their study exposed a surprising segmentation of these networks into three regions: singletons who do not participate in the network; isolated communities that mostly display star structure; and a giant component with a well-connected core region. The authors also presented a simple model of network growth which captures these aspects of component structure. 
In \cite{LKF}, Leskovec et al. examine macroscopic properties of network evolution, like densification and shrinking diameters while in \cite{micro_evol} the authors present a microscopic analysis of the edge-by-edge evolution of four large online social networks. In \cite{micro_evol} Leskovec et al. show that most new edges span very short distances, typically closing triangles and motivated by the observations, they develop a model of network evolution.
There is also a more theoretical brach of research that focus on network evolution. One example is Fenner et al. \cite{stoch_evol}, where they present a stochastic model of network growth, where new actors may join the network, existing actors may become inactive and, at a later stage, may reactivate themselves. Their model captures the evolution of the network, assuming that actors attain new relations or become active according to the preferential attachment rule.
Many of these above studies were performed to understand the growth of the network overall.
On the other hand, in our study we focus on evolution of the 
underlying communities in a network rather than the network itself. 

There is also a large body of work on identifying the communities in 
networks \cite{newman, danon, www10, comm_lin}.
A community of a network or tightly connected clusters is typically thought of
as a group of nodes with more and/or better interactions amongst
its members than between its members and the remainder of the
network \cite{define_comm}. An extensive review of the literature on
community detection can be found in \cite{com_detect}. 
However, our study is different than this body of work since the communities are 
already explicitly identified. As mentioned before, our focus is understanding 
the mechanisms by which the communities grow.

A parallel body of work is concerned with user behavior in online communities.
In \cite{pref_behav}, the authors explore a large corpus of online
communities that vary widely in size, subject matter and privacy and focus on user
engagement. They present a simple model explaining long-term heavy
engagement as a combination of user-dependent and group dependent factors. 
In \cite{dyn_conversations}, Kumar et al. analyze the structure of conversations by considering three different social datasets; Usenet groups, Yahoo! Groups, and Twitter. They propose simple mathematical models for the generation of basic conversation structures.
In another study Goyal et al. \cite{leaders} introduce a frequent pattern mining approach 
to discover leaders from community actions. They focus on identifying which users 
are leaders when it comes to setting the trend for performing various actions.
In \cite{user_grouping}, Shi et al. look at the online forums and study the patterns of user participation behavior, and the feature factors that influence such behavior on different forum datasets.
Similarly, in \cite{groupthink}, Hui and Buchegger present a simple social influence model to 
describe and explain the group joining process of users in online social networks.

A previous study that inspired this work is  Backstrom
et al. \cite{group_formation} paper, where the authors focus on the role of common
friends on a community's formation and growth. They use decision-tree techniques
to identify the properties of the social network that affect its evolution.

%COMPARE WITH OUR RESULTS?????????????????