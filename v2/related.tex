Many studies on online social networks and biological
networks focused on properties of network evolution such as shrinking 
diameters and edge creation \cite{kumar_tomkins, LKF, micro_evol}.
These studies focus on understanding the change in the overall
network over time. On the other hand, the evolution of the 
underlying communities in a network has not been studied as much. 

There is also a large body of work on identifying the communities in 
networks \cite{newman, danon, www10}.
A community of a network or tightly connected clusters is typically thought of
as a group of nodes with more and/or better interactions amongst
its members than between its members and the remainder of the
network \cite{define_comm}. However, in our study, the communities are 
already explicitly identified. As mentioned before, our focus is understanding 
the mechanisms by which the communities grow.

A previous study that inspired this work is  Backstrom
et al. \cite{group_formation} paper, where the authors focus on the role of common
friends on a community's formation and growth. They use decision-tree techniques
to identify the properties of the social network that affect its evolution.

%COMPARE WITH OUR RESULTS?????????????????