The process by which people get together and form communities has been an active area of research, at the intersection of psychology, sociology, and computer science \cite{social_ref}. The recent emergence of online social networks facilitates a data-driven approach to the study of this topic and can lead to exciting new insights about the behavior of individuals. 

Understanding the dynamics of communities is crucial as they give hints about the larger 
social network that they form. However, one challenge comes form the fact that it is difficult to
identify the underlying communities of a network \cite{newman, danon, www10}.  To overcome this challenge, 
we used an online social networking and blogging site, LiveJournal, as our primary source of data.
LiveJournal is not only a large social network but also it has a significant collection of explicitly
identified communities. By analyzing data that we have collected over a span of 5 months, we aim to answer mainly the following question:

\begin{itemize}
\item Can we predict whether a community will grow over time?
\end{itemize}

For studying the growth of communities we investigate several features such as clustering coefficient, and average degree. From our analysis, among the features we considered, the parameter that turns out to have the most significant effect is related to what we call $\emph{attractiveness}$. We define the notion of being attractive for members of a community. It reflects how successful each person has been in attracting his/her friends to join the community.

Using regression analysis, we show that there is a strong positive correlation between the percentage of attractive members of a community and its growth. We also tried to measure the correlation between the growth and some structural properties of the communities such as clustering coefficient and average degree. None of them showed evidence of strong correlation as \emph{percentage of attractive members} did. 

Later, in order to understand the notion of being attractive more, we looked at how it changes with over time or in different communities. Our data shows that among the newly joined members, the percentage of attractive members is much higher. People that are members of multiple communities tend to be attractive in most of them at the same time. Our last observation is that people are attractive for a short span of time.

